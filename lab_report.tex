\documentclass[14pt]{extreport}
\usepackage{fontspec}
\usepackage{polyglossia}
\setdefaultlanguage{russian}
\setotherlanguages{english}
\setmainfont{Tempora}
\usepackage{setspace}
\singlespacing
\usepackage[margin=2cm]{geometry}
\usepackage{titlesec}
\usepackage{titletoc}
\usepackage{color}

\titleformat{\chapter}[display]
{\centering \normalfont\large\bfseries}{}{20pt}{}
\titlespacing*{\chapter}{0pt}{-50pt}{20pt}
\titleformat{\section}{\normalfont\normalsize\bfseries}{\thesection}{1ex}{}

\setlength{\parindent}{0pt}
\setlength\parskip{\medskipamount}


\begin{document}

\begin{titlepage}
  \begin{center}
    \onehalfspacing
    \begin{small}
      \singlespacing
      \textbf{МИНИСТЕРСТВО ЦИФРОВОГО РАЗВИТИЯ,\\
      СВЯЗИ И МАССОВЫХ КОММУНИКАЦИЙ РОССИЙСКОЙ ФЕДЕРАЦИИ}\\
      \bigskip
      \textbf{ФЕДЕРАЛЬНОЕ ГОСУДАРСТВЕННОЕ БЮДЖЕТНОЕ ОБРАЗОВАТЕЛЬНОЕ\\
      УЧРЕЖДЕНИЕ ВЫСШЕГО ОБРАЗОВАНИЯ\\
      «САНКТ-ПЕТЕРБУРГСКИЙ ГОСУДАРСТВЕННЫЙ УНИВЕРСИТЕТ\\
      ТЕЛЕКОММУНИКАЦИЙ ИМ. ПРОФ.М.А. БОНЧ-БРУЕВИЧА»\\
      (СПбГУТ)}
    \end{small}

    \textcolor[RGB]{160,160,160}{\rule{\textwidth}{0.5pt}}

    Факультет: \underline{\smash{Инфокоммуникационных сетей и систем}}\\
    Кафедра: \underline{\smash{Защищенных систем связи}}\\
    Дисциплина: \underline{\smash{Защищенные операционные системы}}\\

    \vspace{3em}

    \textbf{ОТЧЕТ ПО ЛАБОРАТОРНОЙ РАБОТЕ}

    \vspace{4em}

     Настройка маршрутизации и сетевых политик\\
     \vspace{-3ex}
     \rule{\textwidth}{0.5pt}\\
     \vspace{-2ex}
     \begin{footnotesize}
       \textit{(тема отчета)}
     \end{footnotesize}

    \vspace{1em}

    \leftline{Направление/специальность подготовки}

     10.05.02 Информационная безопасность телекоммуникационных систем\\
     \vspace{-3ex}
     \rule{\textwidth}{0.5pt}\\
     \vspace{-2ex}
     \begin{footnotesize}
        \textit{(код и наименование направления/специальности)}
     \end{footnotesize}

    \vspace{3ex}

    \begin{flushright}
      \begin{minipage}{0.5\textwidth}
        Студент:

	\vspace{2ex}

	Песин А.Е., ИБС-03\hfill
	\vspace{-3ex}
	\rule{23ex}{0.5pt}\hfill\rule{10ex}{0.5pt}
	\vspace{-2ex}

        \begin{footnotesize}
	  \hspace{6ex}\textit{(Ф.И.О., № группы)\hfill(подпись)\hspace{3ex}}
        \end{footnotesize}

	Мурзаева А.В., ИБС-03\hfill
	\vspace{-3ex}
	\rule{23ex}{0.5pt}\hfill\rule{10ex}{0.5pt}
	\vspace{-2ex}

        \begin{footnotesize}
	  \hspace{6ex}\textit{(Ф.И.О., № группы)\hfill(подпись)\hspace{3ex}}
        \end{footnotesize}

	\vspace{2ex}

        Преподаватель:

	ст. преп. Пестов И.Е.\hfill
	\vspace{-3ex}
	\rule{23ex}{0.5pt}\hfill\rule{10ex}{0.5pt}
	\vspace{-2ex}

        \begin{footnotesize}
	  \hspace{4ex}\textit{(Ф.И.О. преподавателя)\hfill(подпись)\hspace{3ex}}
        \end{footnotesize}
      \end{minipage}
    \end{flushright}

    \vfill

    г. Санкт-Петебрург\\
    2024
  \end{center}
\end{titlepage}


\renewcommand*\contentsname{\centering \normalsize Оглавление}
\addtocontents{toc}{\vspace{-4ex}}
\tableofcontents


\chapter{Потускнели светлые лики икон}
\section{Цель}
Идейные соображения высшего порядка, а также высокотехнологичная концепция
общественного уклада однозначно фиксирует необходимость стандартных подходов.
Однозначно, непосредственные участники технического прогресса неоднозначны и
будут рассмотрены исключительно в разрезе маркетинговых и финансовых
предпосылок.


\section{Задачи}
\begin{itemize}
  \item Разнообразный и богатый опыт говорит нам, что базовый вектор развития
	играет важную роль в формировании приоретизации разума над эмоциями.

  \item Принимая во внимание показатели успешности, высокое качество позиционных
	исследований, в своём классическом представлении, допускает внедрение
	благоприятных перспектив. 

  \item С другой стороны, убеждённость некоторых оппонентов предполагает
	независимые способы реализации существующих финансовых и
	административных условий.
\end{itemize}


\section{Ход действий}
Противоположная точка зрения подразумевает, что активно развивающиеся страны
третьего мира и по сей день остаются уделом либералов, которые жаждут быть
ассоциативно распределены по отраслям.

Как принято считать, действия представителей оппозиции освещают чрезвычайно
интересные особенности картины в целом, однако конкретные выводы, разумеется,
ассоциативно распределены по отраслям.

В своём стремлении повысить качество жизни, они забывают, что курс на социально-ориентированный национальный проект играет определяющее значение для анализа существующих паттернов поведения.

А ещё многие известные личности набирают популярность среди определенных слоев населения, а значит, должны быть объединены в целые кластеры себе подобных.


\section{Вывод}
Предварительные выводы неутешительны: сплочённость команды профессионалов, а
также свежий взгляд на привычные вещи — безусловно открывает новые горизонты для
переосмысления внешнеэкономических политик. Повседневная практика показывает,
что новая модель организационной деятельности говорит о возможностях
переосмысления внешнеэкономических политик. Не следует, однако, забывать, что
консультация с широким активом обеспечивает широкому кругу (специалистов)
участие в формировании системы массового участия.

\end{document}

